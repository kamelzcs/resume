%______________________________________________________________________________________________________________________
% @brief    LaTeX2e Resume for Kamil K Wojcicki
\documentclass[margin, centered]{resume}


%______________________________________________________________________________________________________________________
\begin{document}
\name{\Huge Jilong Liao}
\address{\\
\textbf{Email}: jilong.liao@outlook.com \hspace{50mm}\hfill \textbf{Address}: 5800 Central Ave. Pike\\
\textbf{Phone}: 865-360-6082 \hfill Apt 704\\
\textbf{Home Page}: http://jilongliao.com \hfill Knoxville TN, 37912
}
\begin{resume}
	%\vspace{-1mm}
    %__________________________________________________________________________________________________________________
    % Contact Information
%    \section{\mysidestyle Contact\\Information}
%	 Center for Intelligent System and Machine Learning(CISML)\\
%    \hfill Phone: (865)-360-6082  \vspace{0mm}\\\vspace{0mm}%
%    EECS Department                          				 \hfill Email: jliao2@utk.edu   \vspace{0mm}\\\vspace{0mm}%
%    The University of Tenneesee, Knoxville TN      				 \hfill Home Page: http://www.jilongliao.info \vspace{0mm}\\\vspace{-4.5mm}%


    %__________________________________________________________________________________________________________________
    % Research Interests
    %\section{\mysidestyle Research\\Interests}

    %Speech processing, speech enhancement, speech and speaker recognition, speech perception, \\ 
    %machine learning and pattern recognition.


    %__________________________________________________________________________________________________________________
    % Education
    \section{\mysidestyle Education}

    \textbf{The University of Tennessee}, Knoxville, TN \hfill Jan 2011 -- Dec 2013\vspace{0mm}\\\vspace{0mm}%
    \emph{Master's in Computer Science} \vspace{-1mm}\\\vspace{-2mm}%
    \begin{list2}
        \item Major GPA: 3.8/4.0
        \item Thesis: \emph{A Privacy-Aware Distributed Storage and Replication Middleware for Heterogeneous Computing Platform}
    \end{list2}\vspace{-1mm}
    \textbf{University of Electronic Science and Technology of China} \hfill Sept 2006 -- Jul 2010\vspace{0mm}\\\vspace{0mm}%
	\emph{Bachelor's in Communication Engineering with Honors} \vspace{-1mm}\\\vspace{-2mm}%
	\begin{list2}
		\item Major GPA: 3.9/4.0
	\end{list2}\vspace{-1mm}

    %__________________________________________________________________________________________________________________
    % Professional Experience
    \section{\mysidestyle Professional\\Experience}

    \textbf{The University of Tennessee}, Knoxville, TN \\\vspace{0mm}%
    \emph{Research Assistant} \hfill Jan 2011 -- present\vspace{0mm}\\
    $\bullet$ Research into projects includes mobile computing, machine learning and distributed system\vspace{-6mm}\\
    %$\bullet$ Promoted to \textsl{Leading Research Assistant}\vspace{-6mm}\\

    \textbf{Microsoft}, Redmond, WA \\\vspace{0mm}
    \emph{Software Development Engineer Intern} \hfill May 2013 -- Aug 2013 \vspace{0mm}\\
    $\bullet$ Project: Component Power Characterization Suite for SoC System\vspace{0mm}\\
    $\bullet$ Designed a generic and scalable framework which built in CPU, memory and GPU component \vspace{0mm}\\
    $\bullet$ Extended the framework to muti-variant and performance analysis metrics support \vspace{-6mm}\\

    \textbf{The University of Tennessee}, Knoxville, TN \\\vspace{0mm}%
    \emph{Teaching Assistant} \hfill Jan 2011 -- Dec 2011\vspace{0mm}\\
    $\bullet$ Taught undergraduate students C++ programming and graded students' homework\vspace{-6mm}\\
	
    \textbf{Ericsson (China) Communications Co.}, Chengdu, China \vspace{0mm}\\%
	\emph{Integration/Verification Engineer Intern} \hfill Dec 2009 -- Mar 2010\vspace{0mm}\\
    $\bullet$ Integrated TD-SCDMA base-station system and fixed one software bug\vspace{-5mm}\\
    %$\bullet$ Improved one verification process by reducing 25\% percent of time via my Python script\vspace{-5mm}\\

    %__________________________________________________________________________________________________________________
    % Computer Skills
    \section{\mysidestyle Professional \\Skill} 
	
	$\bullet$ \textbf{Familiar with}: Software Engineering, Machine Learning, Big Data Tools, Distributed System, Android, MySQL\\
    $\bullet$ \textbf{Strong Language}: C/C++(7 yrs), Java(4 yrs), Python(4 yrs)\vspace{-5mm}\\
	%$\bullet$ \textbf{Community Handler}: little-eyes(TopCoder)\vspace{-2mm}\\

    %__________________________________________________________________________________________________________________
    % Publications
    \section{\mysidestyle Publications}
    Z. Wang, J. Liao, Q. Cao, H. Qi and Z. Wang, ``Achieving k-barrier Coverage in Hybrid Directional Sensor Networks'', \emph{IEEE Transactions on Mobile Computing (TMC)}, 2013, to appear. \vspace{1mm} \\
    J. Liao, K. Lu and Q. Cao, ``Uno: A Privacy-Aware Distributed Storage and Replication Middleware for Heterogeneous Computing Platforms'', \emph{IEEE MASS}, 2013. \vspace{1mm} \\
    Z. Wang, J. Liao, Q. Cao, H. Qi and Z. Wang, ``Barrier Coverage in Hybrid Directional Sensor Networks'', \emph{IEEE MASS}, 2013.\vspace{1mm} \\
    J. Liao, Z. Wang, Q. Cao and H. Qi, ``Smart Diary: the Narrative of Your Daily Life'', \emph{NSF Southeastern Workshop on Cognitive Sensing, Computing and Networking and Their Applications in Human-Cyber-Physical Systems}, Tuscaloosa, Alabama. August 15, 2012. \vspace{1mm} \\
    J. Liao and Q. Cao, ``Demo Abstract: Uno - A Sharing Infrastructure for Smartphone Sensors and Files'', \emph{ACM SenSys}, 2011.\\ \vspace{-5mm}
    
    \section{\mysidestyle Academic Services}
    $\bullet$ Reviewer, IEEE International Conference on Computer Communications (INFOCOM) \hfill 2014\\
    $\bullet$ Reviewer, International Conference on Wireless Sensor Networks for Developing Countries \hfill 2013\\
    $\bullet$ Reviewer, IEEE Journal of Communications and Networks (JCN) \hfill 2013\\
    $\bullet$ Reviewer, IEEE International Conference on Mobile Ad-hoc and Sensor Systems (MASS) \hfill 2012\\
    $\bullet$ Reviewer, IEEE International Conference on Mobile Ad-hoc and Sensor Networks (MSN) \hfill 2011\\ \vspace{-5mm}
%    \section{\mysidestyle Recent \\Research Projects}
	
%	\textbf{Uno: A Novel Cloud Data Privacy Middleware on Smartphones} \hfill Fall 2012\\
%	$\bullet$ Extracted a time slotted usage and status patterns from $22$ users' one-year smartphone usage data.\\
%	$\bullet$ Proposed a novel cloud storage design which separates physical data and meta data.\\
%	$\bullet$ Designed a fine-grained replication algorithm achieving $98\%$ availability from random use smartphones. \vspace{-6mm}\\
	
%	\textbf{Link Quality Prediction in Wireless Sensor Networks} \hfill Fall 2012\\
%	$\bullet$ Achieve 95\% percent accuracy in prediction by using state-space model.\\
%	$\bullet$ Implemented the prediction algorithm robustly on sensor motes.\\
%	$\bullet$ A journal paper about this prediction system is under peer review. \vspace{-6mm}\\
	
	%\textbf{Speculative Smartphone App Prelaunch} \hfill Fall 2012\\
	%$\bullet$ Discovered a unique app usage pattern from both Android and iOS users correlated to timeline.\vspace{0mm}\\%
	%$\bullet$ Designed a novel prelaunch scheduling algorithm boosting the original algorithm by 10\% to 75\%.\vspace{2mm}\\%
	%$\bullet$ Implemented the algorithm in Android OS which reduces 25\% of app launch time. \vspace{3mm}\\%
    %Paliwal, K.K., Shannon, B.J., Lyons, J.G. and K.K. W\'ojcicki,
    %``Speech-signal-based frequency warping'',
    %\textsl{IEEE Signal Process. Lett.}, Vol. 16, No. 4, pp. 319-322, 2009.
%\vspace{-2mm}
%	\textbf{Friendbook: A Semantic Friend Recommendation System in Social Network} \hfill Fall 2011\\
%	$\bullet$ Analyzed the users daily life pattern through smartphones.  \vspace{0mm}\\%
%	$\bullet$ Designed a special ranking algorithm for potential friends improving 30\% accuracy. \vspace{0mm}\\%
%	$\bullet$ Reinforced the similarity measurement improving the generalness by 45\%.\vspace{-5mm}\\
    %Paliwal, K.K. and K.K. W\'ojcicki,
    %``Effect of analysis window duration on speech intelligibility'',
    %\textsl{IEEE Signal Process. Lett.}, Vol. 15, pp. 785-788, 2008.

%\vspace{-2mm}
    %Stark, A.P., W\'ojcicki, K.K., Lyons, J.G. and K.K. Paliwal,
    %``Noise driven short time phase spectrum compensation procedure for speech enhancement'',
    %In \textsl{Proc. INTERSPEECH}, pp. 549-552, 2008.

%\vspace{-2mm}
    %W\'ojcicki, K.K., Milacic, M., Stark, A.P., Lyons, J.G. and K.K. Paliwal,
    %``Exploiting conjugate symmetry of the short-time Fourier spectrum for speech enhancement'',
    %\textsl{IEEE Signal Process. Lett.}, Vol. 15, pp. 461-464, 2008.

%\vspace{-2mm}
    %W\'ojcicki, K.K. and K.K. Paliwal,
    %``Importance of the dynamic range of an analysis window function for phase-only and magnitude-only reconstruction of speech'',
    %In \textsl{Proc. ICASSP}, pp. 729-733, 2007.

    %__________________________________________________________________________________________________________________
    % Honours and Awards
    \section{\mysidestyle Honours and\\Awards} 

    $\bullet$ \textsl{Chancellor's Citations for Extraordinary Professional Promise} \hfill 2013 \\
    $\bullet$ \textsl{EECS Department Excellent Fellowship}, granted to outstanding graduate students \hfill 2012, 2011\vspace{0mm}\\%
    $\bullet$ \textsl{ACM SenSys Student Travel Grant}, granted to top students who published papers\hfill 2011\vspace{0mm}\\%
    $\bullet$ \textsl{Second Prize}, the $8^{th}$ UESTC Programming Contest \hfill 2010\vspace{0mm}\\%
    $\bullet$ \textsl{Outstanding Graduation Honor}, awarded to 10/3,500 students in UESTC \hfill 2010\vspace{0mm}\\%
    $\bullet$ \textsl{National Fellowship}, awarded to top 2\% students in UESTC \hfill 2009, 2008\vspace{0mm}\\%
    $\bullet$ \textsl{Bronze Medal}, the $33^{rd}$ \textbf{ACM-ICPC} Asia Regional Contest (Beijing) \hfill 2008\vspace{-5mm}\\ 

    %__________________________________________________________________________________________________________________
    % Referees
%    \section{\mysidestyle Referees} 
%    {\sl Available on request.}

%______________________________________________________________________________________________________________________
%\section{\mysidestyle Referees} 

%\textbf{Dr. Qing Cao}\\
%Assistant Professor, The University of Tennessee\\
%Email: cao@utk.edu\\
%Home Page: http://web.eecs.utk.edu/$\sim$qcao1/

%______________________________________________________________________________________________________________________
\end{resume}
\end{document}


%______________________________________________________________________________________________________________________
% EOF

